\documentclass[intlimits,twoside,a4paper,11pt]{article}

\usepackage[utf8]{inputenc}
\usepackage[T2A]{fontenc}
\usepackage[english,russian]{babel}

\usepackage[eqsecnum]{kioj4}

\setcounter{page}{60}
\journalsection{empty}

\begin{document}
%вставляем картинку
\noindent\includegraphics[width=7cm]{events.png}
%или вот такой вариант, если надо картинку справа
%\noindent\hfill\includegraphics[width=7cm]{events.png}

%\section* означает, что раздел не нужно нумеровать.
\section*{ХРОНИКИ ПРОЕКТА МЕТАМАТ (METAMATH) 
\\«Современные образовательные технологии в математических учебных программах в инженерном образовании России»:\\ 
семинар на базе Технического Университета Тампере 26--27 июня 2014}

Прошедший на базе Технического Университета Тампере семинар был посвящен обсуждению документа, в котором зафиксированы базовые компетенции по различным предметам курса математики европейских технических вузов. Документ называется «Framework for Mathematics Curricula in Engineering Education», подготовлен он под эгидой SEFI~--- Европейского общества технического образования.

Документ состоит из 87 страниц. Ниже представлено содержание основной части:

\textbf{General Mathematical Competencies for Engineers}\\ 
Competencies, Dimensions, and Clusters.\quad\quad Example.   \quad\quad Profiles.

\textbf{Content-related competencies, knowledge, and skills.}\\
Core Zero.  \quad\quad  Core Level 1.  \quad\quad  Level 2. \quad\quad  Level 3.

\textbf{Teaching and learning environments.}\\
Teaching and learning arrangements.  \quad\quad  Transition issues.  \quad\quad  Mathematics technology.  \quad\quad  Integrating the mathematics curriculum into the engineering study course.  \quad\quad  Attitudes.

\textbf{Assessment.}\\ 
Forms of assessment.  \quad\quad  Requirements for passing.  \quad\quad  Assessing competencies.  \quad\quad  Technology-supported assessment.

В этом документе представляет интерес попытка конструктивно определить конечные результаты обучения (Learning outcomes) в форме достаточно подробно сформулированных умений, которые и занимают большую часть этого документа. Результаты сформулированы на трёх уровнях: 0-уровень соответствует школьным знаниям, 1 и 2 уровни~--- обучению на младших курсах вуза, 3 уровень предполагает специализацию и реализуется на последних курсах обучения.

Вот, например, какие результаты ожидаются при изучении темы «Графа» курса дискретной математике на уровне 1:\\
Graphs.
As a result of learning this material you should be able to\\
-- recognise an Euler trail in a graph and / or an Euler graph (узнавать Эйлеровы пути в графе и/или Эйлеровы графы),\\
-- recognise a Hamilton cycle (path) in a graph (узнавать Гамильтоновы графы),\\
-- find components of connectivity in a graph (находить компоненты связности графа),\\
-- find components of strong connectivity in a directed graph (и компоненты сильной связности),\\
-- find a minimal spanning tree of a given connected graph (находить минимальное остовное дерево).

Этот документ используется и для создания так называемого руководства студента (students guide), с помощью которого студент сможет самостоятельно определять способ изучения материала, например, разобраться с материалом самостоятельно, используя интернет-ресурсы.

На семинаре была представлена методика сравнения курсов в разных странах и на её основе произведено сравнение нескольких курсов российских вузом со стандартом SEFI. Предварительный вывод состоит в том, что на первых курсах российских технических вузов математика изучается с большей глубиной и в более широком диапазоне, покрывая все 3 уровня программы SEFI.

\end{document}
