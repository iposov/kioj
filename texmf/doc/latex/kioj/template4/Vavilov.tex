% Пример статьи для журнала КИО
\documentclass[intlimits,twoside,a4paper,11pt]{article}

% текст статьи рекомендуется писать в кодировке utf-8
\usepackage[utf8]{inputenc}

% Подключение пакета с шаблоном журнала.
% Необзятальная опция eqsecnum указывает, что в каждом разделе формулы нумеруются заново.
\usepackage[english,russian]{babel}
\usepackage{graphicx}
\usepackage{amssymb}
%\usepackage{hyperref}


\usepackage{amsmath}
\usepackage{kioj4-monographie}
% Здесь можно подключить свои пакеты. Некоторые пакеты работают только если их подключить перед предыдущей строкой \usepackage[eqsecnum]{kioj4}
% \usepackage[•]{•}
\usepackage{array}


% Указываем номер страницы, с которой начинается статья, можно не указывать, будет нумероваться с первой.
\setcounter{page}{5}

%рубрика в журнале, обычно не известна при подаче статьи, поэтому не указываем
\journalsection{algorithmic-mathematics}
%\journalsectionnoimage{informatics}
%\journalsection{informatics}
%\journalsection{information-systems}
%\journalsection{software-engineering}
%\journalsectionnoimage{computer-in-education}
%\journalsection{popular-science-articles}
%\journalsection{programming-practice}
%\journalsection{editorial-column}
%\journalsection{specialists-training-new-teaching-methods}
%\journalsection{specialists-training-training-programs}
%\journalsection{specialists-training-professional-standards}
%\journalsection{open-problems-for-young-scientists}
%\journalsection{empty}

%указывам год, номер выпуска неизвестен, поэтмоу ставим прочерк, указываем edu, что означает журнал "компьютерные инструменты в образовании".
\issue{2020}{2}{edu}
% Указываем УДК для статьи
\udknumber{004 + 003 + 51 + 510/512}
% Указываем DOI для статьи, можно не указывать
\doinumber{10.32603/2071-2340-2020-2-5-26}

%Указываем название статьи в обычном регистре в квадратных скобках, потом заглавными буквами в фигурных скобках. Второй вариант будет использоваться только в заголовке, первый вариант будет использоваться во всех других местах, в частности, при цитировании.

\title[Компьютер как новая реальность математики]{КОМПЬЮТЕР КАК 
	НОВАЯ РЕАЛЬНОСТЬ МАТЕМАТИКИ}

%%%%%%%%%%%%%%%%%%%%%%%%%%%%%%%%%%%%%

\usepackage{amssymb}
%% \usepackage{amsmath}
%% \usepackage{mathrsfs}

\newcommand\SL{{\rm SL}}
\newcommand\GL{{\rm GL}}
\newcommand\PSL{{\rm PSL}}
\newcommand\FG{{\rm FG}}
\newcommand\Int{{\Bbb Z}}
\newcommand\Nat{{\Bbb N}}
\def\A{\operatorname{A}}
\def\B{\operatorname{B}}
\def\C{\operatorname{C}}
\def\D{\operatorname{D}}
\def\F{\operatorname{F}}
\def\G{\operatorname{G}}
\def\E{\operatorname{E}}
\def\K{\operatorname{K}}
\def\GF#1{{\mathbb F}_{\!#1}}



%%%%%%%%%%%%%%%%%%%%%%%%%%%%%%%%%%%%%


%Осмысление задачи посредством моделирования предметной области (с целевой установкой на создание генератора задач)

%О цифровизации школьного математического образования в рамках проекта ReMath и за его пределами
%Определение совокупности коллекций для баз данных типа ключ-документ по заданному набору свойств объектов и запросов к базе данных

\begin{document}
	\maketitle

	\section{ВВЕДЕНИЕ}
	
	Мы являемся свидетелями и участниками одной из наиболее масштабных и 
	значимых метаморфоз, которые математика проходила за всю свою 
	многотысячелетнюю историю. В 1993 году Дорон 
	Зайльбергер\footnote{~\url{https://sites.math.rutgers.edu/\~zeilberg/}} \cite{Ze} 
	сравнил происходящую в
	математике революцию с научной революцией XVI--XVII веков: 
	``The computer has already started doing to mathematics what the 
	telescope and microscope did to astronomy and biology.''
	
	Значительно расходясь с ним в оценке смысла и {\it целей\/} этой революции,
	я полностью согласен с такой оценкой ее содержания и {\it значения\/}. 
	Ключевое слово здесь, однако, ``started''. Как всегда, {\it настоящая\/}
	революция приходит незаметно, <<лучшие и прочнейшие изменения суть те, 
	которые происходят от улучшения нравов, без всяких насильственных 
	потрясений>>. 
	\par
	Подлинный масштаб {\it произошедших\/} изменений --- и, тем более,
	того, что неминуемо произойдет в ближайшее время, ---  с трудом осознается 
	современниками. Независимо от всяких компьютерных доказательств
	появление компьютеров {\it уже\/} изменило нашу жизнь как математиков
	и наше восприятие математики. Это касается самых базовых представлений,
	гораздо более глубоких и важных, чем любые теории: контакт с математической
	реальностью, роль эксперимента, баланс идей и вычислений,
	соотношение большого и маленького --- и проблема {\it промежуточных\/} 
	размеров, о которой мы ранее не задумывались, --- конечного и бесконечного, 
	случайного и детерминированного, доказуемого и недоказуемого, вычислимого 
	и невычислимого\footnote{~Конечно, здесь происходит сознательная отсылка 
		к названию книг Юрия Ивановича Манина \cite{M1,M2}. Как мне кажется,
		на них очень интересно взглянуть и сегодня, чтобы оценить, насколько изменилась 
		наша переспектива за прошедшие 40 лет. Вот ссылка на личную страницу Юрия
		Ивановича в Бонне \url{https://www.mpim-bonn.mpg.de/node/99}, где можно найти
		некоторые его дальнейшие тексты общего характера.}, возможного и невозможного... 
	И, что самое главное, интересного и неинтересного, куда вообще мы смотрим, 
	и на что мы при этом обращаем внимание.
	\par
	Среди прочего, эта революция дает нам возможность преодолеть 
	катастрофический и постоянно расширяющийся начиная с XIX века отрыв практикующих 
	математиков от других образованных слоев общества и от образования в целом.
	И сегодня уровня математики XVIII века достигают единицы,
	а классическая математика XIX века остается исключительным достоянием 
	профессиональных математиков и наиболее продвинутых теоретических физиков.
	В этом смысле мало что изменилось (а если изменилось, то только в худшую сторону) 
	за более чем 60 лет с тех пор как Гуго Штейнгауз произнес свою знаменитую диатрибу: 
	``Tote\.{z} w niej bez por\'ownania wyra\'zniej ni\.{z} w innych dyscyplinach
	wyst\c{e}puje rozci\c{a}ganie si\c{e} pochodu ludzko\'sci.
	W tej chwili \.{z}yj\c{a} 
	r\'ownocze\'{s}nie na ziemi ludzie, kt\'orzy matematicznie 
	nale\.{z}\c{a} do epoki starszej od piramid egipskich, i ci s\c{a} w znacznej 
	wi\c{e}kszo\'sci; niewielki odsetek dotar\l{} do \'sredniowecza, a do XVIII w. nie 
	doszed\l{} ani jeden cz\l{}owiek na tysi\c{a}c.''\footnote{~<<Также в ней несравненно
		отчетливее, чем в других науках, проявляется то, как растянут прогресс
		человечества. Сегодня одновременно с нами на земле живут люди, с точки
		зрения понимания математики принадлежащие эпохе более древней, чем 
		египетские пирамиды, и их значительное большинство; небольшой процент
		добрался до средних веков,  до XVIII века не дошел и один из тысячи>>,
		цитируется по \cite[p. 32]{St}; первый русский перевод в \cite[c. 375]{St1}. }
	\par
	
	Школьные курсы математики отстают от современных потребностей на
	тысячелетия. Использование школьной математики как барьера и фильтра
	и возникновение в~XIX веке отдельной <<экзаменационной математики>>
	также не добавили ей общественных симпатий. Отрыв современной математики 
	от общества и образования наносит чудовищный вред обеим сторонам. 
	Математика является {\it важнейшей\/} частью человеческой культуры. Это включение
	нужно читать в обе стороны --- без математики человеческая культура неполна.
	Более того, Освальд Шпенглер писал --- и это, очевидно,
	верно, если рассматривать масштабы 5--10 столетий, как это делал он, --- 
	что уровень цивилизации определяется уровнем ее математики. Сам
	Шпенглер в своей книге с~полным пониманием обсуждал современные ему теорию 
	множеств и теорию групп\footnote{~Другой вопрос, что русские переводы
		осуществлены, конечно, без какого-либо понимания содержания. Так, 
		<<трехмерный континуум>>
		переведен как <<непрерывность трех измерений>>, как будто в 1923
		году уже пользовались \ {\tt Google translate}. С современными переводами я не
		сравнивал.}. Кто из нынешних 
	гуманитариев слышал о том, что {\it сегодня\/} происходит в математике, даже
	не на уровне понимания, а просто на уровне слов?
	
	Нынешняя компьютерная революция в математике побуждает 
	заново осмыслить наше отношение к математической реальности, к
	истории и практике математики. Во многие периоды истории математика была 
	{\it чрезвычайно\/} успешна в естествознании, первоначально в астрономии и 
	физике, потом в других науках и инженерной практике. 
	Я~глубоко убежден, 
	что сегодня математика могла бы сыграть такую же роль в биологии и науках 
	о человеке, что сегодня у нас есть все необходимые для этого инструменты.
	
	Для этого нужно ровно одно, чтобы математику знали и понимали те, кому 
	предстоит ее применять. Нужно снова поместить центральные современные  
	области математики (= core mathematics) в фокус образования на всех 
	уровнях. Для этого мы должны вернуть широким народным массам понимание 
	того, что реальность говорит с нами языком математики, что математика~--- это 
	единственный и непреложный посредник между духом и материей.
	Компьютеры дают нам возможность показать, что настоящая математика~--- это 
	нечто глубоко вплетенное в реальность, живое, вызывающее восхищение и~любопытство --- mathematics is fun. Будет удивительно, если мы хотя бы не 
	попытаемся воспользоваться этой возможностью.
	\par
	Однако продолжу цитату из Зайльбергера:
	``I can envision an abstract of a paper,
	c.~2100, that reads, "We show in a certain precise sense that
	the Goldbach conjecture is true with probability larger than
	0.99999 and that its complete truth could be determined with
	a budget of \$10 billion."\,''
	\par
	Это показывает, как опасно делать прогнозы, особенно на 107 лет
	вперед. В 2006--2014 годах замечательный перуанский математик
	Харальд Хельфготт\footnote{{~\url{https://webusers.imj-prg.fr/\~harald.helfgott/anglais/}}
		Осенью 2014 года Харальд занимал кафедру Ламе 
		в Петербурге \url{https://chebyshev.spbu.ru/harald-helfgott/}.} 
	полностью решил {\it нечетную\/}\footnote{~Или Зайльбергер все же имел в виду
		более трудную {\it четную} проблему Гольдбаха?} проблему Гольдбаха 
	\cite{H1, H2, H3, H4, HP}. Причем решил не в духе компьютерных доказательств или 
	semi-rigorous mathematics, а в самом классическом смысле (насколько мы
	вообще можем быть уверены в правильности наших доказательств!). 
	\par
	Его доказательство, о котором я расскажу чуть больше в одной
	из следующих частей, состояло из {\it математической\/} проверки 
	\cite{H1,H2,H3}\footnote{~Как еще одно свидетельство времени, Харальд вообще
		не подавал эти статьи в журналы, они существуют только в виде препринтов на \,
		{\tt arXiv}, а собрал их в книгу \cite{H4}, которая должна выйти --- или 
		уже вышла?~--- в Princeton Univ. Press.}
	результата для достаточно больших чисел --- которая {\it самым\/} существенным 
	образом использовала {\it компьютеры\/}! --- и {\it компьютерной\/} проверки 
	\cite{HP} для чисел меньших $8\cdot 10^{30}$ --- которая была бы невозможна без 
	принципиальных {\it математических\/} улучшений имеющихся алгоритмов!
	\par
	Вот что сам Харальд неизменно говорит во введениях к своим работам:
	``The present work would most likely not have been possible without free and publicly available software: 
	{\tt PARI}, {\tt Maxima}, {\tt Gnuplot}, {\tt VNODE-LP}, {\tt PROFIL/BIAS}, 
	{\tt SAGE}, and, of course, {\tt \LaTeX}, {\tt Emacs}, the {\tt gcc} compiler and {\tt GNU/Linux} 
	in general. Some exploratory work was done in {\tt SAGE} and {\tt Mathematica}. Rigorous calculations 
	used either D.~Platt’s interval-arithmetic package (based in part on {\tt Crlibm}) 
	or the {\tt PROFIL/BIAS} interval arithmetic package underlying {\tt VNODELP}'', (см. \cite[p.~7]{H2}, 
	\cite[p.~8]{H3} и \cite[p.~ix]{H4}).
	\par
	В моем случае сам список продуктов сильно отличается от того, что перечисляет
	Харальд (хотя есть и неминуемое пересечение, {\tt Mathematica}, {\tt\TeX}, {\tt Emacs}).
	Кроме того, я старше Харальда и в моем случае многие из аналогичных продуктов 
	использовал не я сам, а мои коллеги и ученики. Но я вполне разделяю его общий 
	пафос в отношении того, что огромная часть {\it фантастического\/} прогресса в 
	математике за последние 30 лет, когда одна за другой решаются классические
	проблемы, открытые до этого за {\it несколько столетий\/}, была бы {\it невозможна\/} без 
	компьютеров (есть, конечно, и другие внутренние причины).
	\par
	В настоящей серии статей я рассказываю об использовании компьютеров в некоторых
	близких мне областях алгебры, а также теории чисел и комбинаторики. Некоторыми 
	из этих вещей я занимался сам, за другими имел возможность наблюдать с близкого
	расстояния.  Как мне кажется, алгебра и теория чисел в этом отношении просто
	слегка опережали некоторые другие области математики (хотя сегодня, может быть, 
	даже и это уже неверно), но сами возникшие при этом феномены носят совершенно 
	общий характер и типичны для всей математики. На этом фоне я выскажу 
	несколько общих соображений о математике и ее преподавании. 
	
	
	%%%%%%%%%%%%%%%%%%%%%%%%%%%%%%%%%%%%%%%%%
	
	\section{ANAMNESIS}
	
	Мой ранний опыт общения с компьютерами был исключительно негативным.
	
	В 30-й школе, где я учился в 9-м и 10-м классах --- Олег Иванов тоже учился там же,  
	но на класс старше меня --- и которая {\it тогда\/} располагалась 
	там же, где сегодня, на углу 7-й~линии и Среднего проспекта, стоял теплый ламповый 
	{\tt Урал-1}, который занимал целую комнату. Данные в него вводились даже еще не с 
	перфокарт, а с перфоленты, то есть фотопленки с дырочками. Неправильно пробитые 
	дырочки нужно было заклеивать маленькими квадратиками при помощи ацетона,
	и они, естественно, при любом перемещении ленты снова отваливались и давали 
	ту же самую ошибку, которую, как тебе казалось, ты только что исправил.   
	
	Кроме того, кодирование тогда происходило даже еще не в языке ассемблера, 
	а просто в~адресах, что, конечно, при любом изменении программы сразу приводило 
	к ошибкам в распределении памяти. Необходимость запоминать трехзначные коды
	операций доставляла отдельно.  В результате к моменту окончания школы мне 
	не удалось пропустить ни одну программу.
	
	На математико-механическом факультете меня продолжили учить тому, что тогда называлось <<программированием>>, 
	то есть кодированию в языке ассемблера, а потом на {\tt Algol-60}. Происходило это примерно 
	таким же образом, как в школе, только данные в~теплую транзисторную {\tt БЭСМ-3}, или 
	как она тогда называлась, вводились с бумажных перфокарт, и студентов к ней не 
	подпускали. Нужно было сдать в окошечко на третьем этаже текст программы и дня
	через три получить пачку карт с прямоугольными дырочками. После этого пачку перфокарт следовало
	(не перепутав порядок) перенести в собственно ВЦ и, тоже дня через три, получить 
	обратно распечатку, в которой обычно сообщалось, что в некоторой строке вместо точки
	с запятой стоит просто запятая. Некоторые распечатки имели более криптический
	характер, с ними полагалось идти к преподавателю, которому обычно после получаса
	усилий удавалось обнаружить причину ошибки.
	
	В целом, по моим воспоминаниям, процесс вычисления на суперкомпьютерах того 
	времени был значительно менее увлекательным и, главное, менее эффективным, 
	чем такие же вычисления на арифмометре {\tt Феникс}. В результате снова за два года я не смог
	пропустить ни одной программы и едва не был уволен со второго курса. Получив все 
	же чудесным образом зачет, я постарался забыть все это, как страшный сон, с твердым
	намерением никогда больше не иметь дела со всем этим <<программированием>>.
	
	Однако, как хорошо известно, ourselves we do not owe. К моменту моего окончания 
	мат-меха в 1974 году наступил первый острый пароксизм 
	компьютерной
	грамотности. Математикам добавили вторую специальность в диплом и штук 
	5--7 курсов по языкам программирования, трансляторам, операционным системам,
	генеративным грамматикам\footnote{~Название этого курса
		служило неиссякаемым источником каламбуров.} и т.\,д. 
	Читались эти курсы ровно в том же стиле, как курсы марксистско-ленинской философии
	и научного коммунизма, и отношение к ним у студентов математиков было примерно 
	такое же. В результате всех этих починов и инноваций мы учились не пять лет, 
	а~пять с половиной, и получили дипломы, кажется, в конце ноября. К счастью, учить нас 
	<<практическому>> программированию никто больше не пытался.
	
	В конце 1980-х годов Партия и Правительство в очередной раз серьезно взялись за 
	борьбу с трезвостью
	и компьютерной грамотностью\footnote{~Джордж Карлин когда-то спросил:
		``If crime fighters fight crime, and fire fighters fight fires, then what do freedom fighters fight?''
		\cite{C}. Сама форма локуции подразумевает только, что нечто является объектом
		борьбы. А вот for или against, это уже вопрос интерпретации.}. Всех преподавателей~мат-ме\-ха обязали пройти курсы
	повышения компьютерной неграмотности. Мы с~Лизой~Дыб\-ковой\footnote{~\url{http://www.mathnet.ru/php/person.phtml?option\_lang=rus\&personid=33393}} 
	учились в группе, 
	которой Олег Граничин\footnote{~\url{https://www.math.spbu.ru/user/gran/oleg\_gr.html}}
	рассказывал основы программи\-рования на {\tt Basic}, и у меня 
	даже где-то сохранилась тетрадка с описанием команд и~ос\-нов синтаксиса, а также 
	несколькими простейшими программами на тему вычисления простых. 
	Программирование на {\tt Basic} понравилось
	мне несколько больше, чем кодиро\-вание в адресах и в языке ассемблера, и 
	гораздо больше, чем программирование на {\tt Algol-60}, но все равно я был крайне 
	удручен 
	соотношением усилий, которые~нужно было вложить в написание простейшей программы 
	на низкоуровневом языке, и~по\-лучающегося результата.
	\par
	Конечно, до нас уже с 1970-х годов доходили слухи, что кто-то где-то что-то считает 
	на компьютере. Иногда даже не слухи, а какие-то реальные продукты этого счета. 
	Так, еще в самом начале 1970-х Марк Иванович 
	Башмаков\footnote{~\href{https://ru.wikipedia.org/wiki/Башмаков\_Марк\_Иванович}{https://ru.wikipedia.org/wiki/Башмаков\_Марк\_Иванович}}
	привез из Парижа три килограмма эллиптических кривых. В середине 1980-х
	появился ATLAS конечных групп \cite{CCNPW}, совершенно фантастический 
	источник знаний, абсолютно необходимый каждому, кто как-то соприкасается с 
	конечными группами в своей работе. Но все это казалось чем-то бесконечно 
	далеким и совершенно недостижимым, чем могут заниматься только специально 
	обученные люди.
	\par
	Впрочем, наступали иные времена, мне выдали два паспорта, синий и красный,
	one for weekdays, one for holidays, и жизнь сразу начала налаживаться.
	
	%%%%%%%%%%%%%%%%%%%%%%%%%%%%%%%%%%%%%%%%
	
	\section{CITT\`A STUDI}
	
	Первый раз мы с Олей прилетели в Милан в октябре 1990 года. Лино Ди 
	Мартино\footnote{~\url{https://www.unimib.it/lino-giuseppe-martino}} встречал нас в Мальпенсе. 
	Мы с Лино довольно долго до этого
	переписывались, но встретились первый раз. Что делают математики, когда встречаются? 
	Естественно, рассказывают друг другу теоремы (``Tell me a Theorem'', как говорил
	в таких случаях Билл Кантор\footnote{~\url{https://pages.uoregon.edu/kantor/}}).
	
	Лино с Кьярой Тамбурини\footnote{~\url{http://www.dmf.unicatt.it/personale/info/Tamburini.html.it}}
	как раз заканчивали обзор \cite{DT} по
	$2$-порождению, и он был весь в этом. С Кьярой мы уже познакомились в Киото 
	и потом вместе кормили священных оленей в Наре. 
	Собственно, познакомил нас Джон 
	Уилсон\footnote{~\url{https://people.maths.ox.ac.uk/wilsonjs/}}  во дворе синтоистского храма. 
	С самим Джоном я к тому времени несколько раз встречался на конференциях.
	По дороге в город Лино рассказал 
	мне ее свежий результат \cite{T}, который они планировали включить в обзор с 
	наброском доказательства.
	А именно, Кьяра доказала, что все специальные линейные группы $G=\SL(n,q)$ 
	степеней $n\ge 25$ над любым конечным полем $\GF{q}$ из $q$ элементов 
	$(2,3)$-порождены.
	Иными словами, в $G$ найдутся элементы $x$ и $y$ порядков 2 и 3 соответственно,
	такие, что $G=\langle x,y\rangle$, подробнее ниже.
	
	Тогда это показалось мне {\it настолько\/} контринтуитивным, что я тут же, по 
	пути из Мальпенсы в Милан,
	заявил, что готов построить контрпример. Post factum мое интуитивное представление 
	о возможном и невозможном в комбинаторной теории групп поменялось на прямо 
	противоположное --- возможно все, что явно не запрещено. Тогда же вместо 
	контрпримера мне удалось за пару вечеров, прежде чем мы с Олей отправились дальше 
	по Италии, в Рим, Флоренцию, Венецию и Падую, придумать новую идею доказательства 
	и реализовать ее от руки --- без всякого компьютера, просто перемножая матрицы
	$5\times 5$ на бумажке --- для группы $\SL(5,q)$. 
	
	Было ясно, что на этом пути получится не только передоказать в большинстве 
	случаев результат Кьяры, но и рассмотреть почти все остававшиеся случаи групп 
	небольших рангов. Было ясно также, что это довольно серьезное предприятие, и
	сделать это вот так от руки за пару вечеров уже не получится, потому что случаи 
	небольших степеней требовали индивидульного анализа.
	Мы подали на совместный проект Consiglio Nazionale delle Ricerche 
	(C.N.R.) и получили его. 
	
	В результате ровно через год после первого визита, осенью 1991 года, я прилетел 
	в~Милан во второй раз, теперь не на неделю, а на 4 месяца. За прошедший год во 
	внешнем мире довольно многое изменилось и продолжало меняться. Помню, что
	тогда перелет Петербург---Милан---Петербург (в оба конца) стоил дешевле, чем
	проезд автобусом Malpensa---Milano Centrale (solo andata), и многим (в том числе и 
	мне) {\it казалось\/}, что все это надолго.
	
	Лино тогда работал еще
	не в Bicocca, а в старом университете в квартале Citt\`a Studi. Он дал мне 
	связку ключей: от входной двери, от коридора на втором этаже, от офиса для
	гостей, от своего офиса, от компьютерной лаборатории, от ксерокса, от кофейной 
	машины, от каких-то еще факультетских 
	помещений\ldots \, --- всего штук 8--10, кажется. Ключ от исторической библиотеки в
	Citt\`a Studi не давали даже профессорам.
	Потом, уже в Билефельде, мне объяснили, 
	что настоящий хозяин --- это не тот, у кого много ключей, а тот, у кого всего один 
	ключ, но зато такой, который открывает {\it все\/} двери. И математическая 
	библиотека в Билефельде никогда не закрывалась на ночь, по крайней мере, в 
	1990-х годах\footnote{~Ну, на самом деле закрывалась, конечно, один раз в году, 
		в ночь с 24-го на 25-е декабря.}, но об этом в следующих частях.
	
	Сам Лино работал дома и приезжал в университет только на лекции. В эти дни 
	мы с ним шли в пиццерию рядом с факультетом (на via Sandro Botticelli, если быть
	совсем точным) и брали за обедом\footnote{~Pranzo, в
		Италии обычно между полуднем и часом дня.} на двоих бутылку 
	чего-нибудь простого, типа ``Bianco di Custoza''. В первый раз я его спросил: 
	<<Лино, как ты собираешься после этого читать лекцию? Меня за такое сразу
	выгнали бы из университета>>. На что он возразил: <<Что ты, мы всегда так
	делаем, от этого качество лекции может только улучшиться>>. В Италии, 
	да и в некоторых других странах, я имел возможность оценить, насколько глубокий 
	смысл имеет эта старинная университетская традиция. 
	
	%%%%%%%%%%%%%%%%%%%%%%%%%%%%%%%%%%%%%%%
	
	\section{{\TeX} AND FRIENDS}
	
	Лино тогда только что, чуть не прямо перед моим приездом, купил в офис новенький 
	{\tt Mac}, 
	вот не допотопный моноблок, а настоящий настольный компьютер: мозги отдельно,
	великолепный по тем временам цветной экран отдельно,
	{\tt LaserWriter} отдельно\footnote{~Ну и совсем отдельно, gratis ---
		завораживающий скринсейвер {\tt Fish}, на который можно было смотреть часами.}
	и~поставил туда лицензионную {\tt Mathematica}\footnote{~Не помню, которую версию, 
		в наших статьях это не упомянуто. Конечно, это легко восстановить по нашим архивам.
		Думаю, что {\tt 2.0}. Во всяком случае в Билефельде я потом продолжал досчитывать в 
		{\tt 2.0}, 
		это как раз, как и дальнейший рост до {\tt 2.2}, задокументировано в других моих статьях.
		Но, конечно, это была совсем другая {\tt Mathematica} под {\tt UNIX}, и нужно отдельно 
		исследовать синхронизацию версий под разные платформы.}
	за какие-то совершенно нереальные деньги типа четырех месячных профессорских 
	зарплат после вычета налогов\footnote{{~\it Итальянского\/} профессора, я имею в виду. 
		Зарплата российского профессора в 1991 году была по обменному курсу меньше одной
		бумажки с портретом Бернини. Так что в тогдашних зарплатах {\it российских\/} 
		профессоров это ближе к 400.}.
	Разумеется, купил все это добро Лино не на свои (на свои он купил себе что-то в
	таком же духе домой), а на грант M.U.R.S.T. по 
	{\it некоммутативной\/} алгебре (есть и такая).
	\par
	Я к тому времени уже больше года как привез из Америки %% оригинальный 
	{\tt IBM PS/2} горизонтальной компоновки под {\tt OS/2} (на тот момент high end в жанре PC,
	который позиционировался как альтернатива Mac, потенциально более совместимая 
	с наличной в России техникой) 
	и \,{\tt HP~LaserJet} \, к нему. Но
	использовал это хозяйство главным образом как продвинутую пишущую машинку,
	типа {\tt IBM}'овских же машинок с шариком\footnote{~Но, как мне напомнил Андрей Семенов,
		именно на этом десктопе \,{\tt model 30\/} \,он произвел основную часть вычислений и 
		изготовил окончательный текст первой в Петербурге диссертации по математике 
		\cite{Se}, полностью подготовленной и распечатанной на {\tt PC}.}. 
	
	Смешно сказать, в качестве основного редактора я тогда использовал {\tt ChiWriter} 
	ровно потому, что он поддерживал русский и позволял хоть как-то набирать формулы. 
	И~с~тем и с другим у всех редакторов семейства {\tt Word\/} и~тогда было совсем плохо, 
	и~теперь много лучше не стало.
	Те, кто не помнят, как выглядели
	моноширинные шрифты и~формулы с индексами в них, могут взглянуть на \cite{V90},
	которая целиком, все 117 страниц Kind und Kegel, вместе с формулами и {\it картинками\/}, 
	набрана в 
	{\tt ChiWriter}\footnote{~Дополнительный дивертисмент состоял в том, что набирал 
		я этот текст в 1991 году в Warwick University. {\it Все\/} компьютеры у них
		оказались {\it с левосторонним движением\/} --- зачеркнуто, {\it другой системы\/}.
		Единственный на весь математический факультет {\tt PC-compatible}, на котором можно 
		было установить {\tt ChiWriter}, стоял в общей комнате, где я это все и печатал.}.
	\par
	В конце 1990-х я уже начал получать от американских коллег первые статьи,
	набранные в {\tt \TeX}. Помню ощущение глубочайшего шока, который я испытал,
	впервые увидев {\tt dvi} файл --- как так, статья еще не вышла, а автор присылает
	мне типографский оттиск. С первого взгляда было ясно, что это продукт {\it совершенно\/} другого
	качества, чем все, что можно было получить при помощи обычных текстовых
	редакторов. 
	
	Ясно, что мы с Лино решили писать свои статьи в \,{\tt plain \TeX}. \,Собственно говоря,
	{\tt AMS-\TeX} к тому моменту уже появился, но был еще эзотерическим продуктом\footnote{~На {\tt AMS-\TeX} я перешел уже потом в Билефельде. И замечательную
		книгу Майкла Спивака \cite{Sp} прочел уже, кажется, только там. Зато потом 
		еще лет 20 продолжал использовать его и по-русски. Для этого Андрей Семенов и Леша
		Степанов раз 8--10 переустанавливали мне русский {\tt AMS-\TeX} на разные компьютеры, 
		но года 4 назад я все же
		неохотно переключился на {\tt \LaTeX} под давлением молодых соавторов.}.
	Напомню, что в те годы инсталляция {\tt \TeX} была творческим процессом, которым 
	могли заниматься только специально обученные люди. Нетривиальным моментом
	было прописывание шрифтов и их согласование с принтером, что все равно
	было огромным прогрессом по сравнению с предыдущим состоянием. Скажем,
	{\tt ChiWriter} требовал загрузки шрифтов в~принтер в каждой сессии.
	
	Поэтому на всем Dipartimento di Matematica {\tt \TeX} был установлен на 2 или 3 
	компьютерах. Вот один как раз в комнате с ксероксом,
	напротив офиса Лино. Процесс изготовления страницы выглядел следующим образом:
	
	1.~Набрать текст на {\tt Mac} в туземном редакторе (вот честно не помню, что-то
	типа {\tt WordPerfect} или {\tt WordStar}, для меня они все на одно лицо).
	
	2. Экспортировать на дискету 3.5'' в формате, читаемом {\tt PC}, ну, допустим, 
	{\tt txt}\footnote{~Экспорт в {\it какие-то\/} неявные форматы, условно {\tt doc} 
		или {\tt rtf} --- я не помню, как они фактически назывались в то время,---
		тоже был, но работал еще гораздо хуже даже между версиями {\it одного и того же\/} 
		редактора для {\tt Mac} и для {\tt PC}.}.
	
	3. Перенести дискету в другую комнату, вставить в {\tt PC}, прочесть и получить 
	сообщение об инвалидных характерах\footnote{~Как известно, даже при экспорте в 
		формате {\tt text only} все слишком умные редакторы семейства {\tt Word} вставляют 
		много отсебятины. Ну, типа, свои форматные символы, которые не отображаются 
		как {\tt ASCII}.}. 
	
	4. Убрать все инвалидные характеры, от{\tt \TeX}овать, получить 
	сообщение об ошибке. 
	
	5. Принять волевое решение, править ли на месте (и потом
	не забыть перенести в~исходный файл) или плюнуть и сразу редактировать снова
	на {\tt Mac}. 
	\par
	Через несколько дней я понял, что процесс, конечно, сходится, но со скоростью
	сходимости придется что-то делать. Я распечатал (или отксерокопировал?) книгу Кнута
	``The {\TeX}book'' и стал ее вечерами читать. В будние дни я проводил часов
	11--12 в офисе Лино за его компьютером, а вечерами читал Кнута. Для меня
	открылся новый мир --- до этого я никогда не задумывался о различии дефиса,
	короткого тире, минуса и длинного тире или о~различии французской и 
	американской типографских точек\footnote{~Позже, вернувшись в конце 
		1990-х в Россию, я с изумлением обнаружил, что все это сокровенное знание
		держат в секрете и от выпускников факультета журналистики СПбГУ.}.
	
	Поскольку дома компьютера у меня не было, пришлось разбирать код. 
	К концу месяца я научился визуализировать {\tt TeX}овский исходник с листа, 
	то есть сразу видеть по {\tt tex}-файлу, как будет выглядеть страница в {\tt dvi}. 
	После этого я стал просто 
	распечатывать исходник в офисе Лино, {\tt Mac} to {\tt Mac}, вечером его 
	править, а {\tt TeX}овать потом уже только окончательную версию, что 
	катастрофически ускорило весь процесс.
	
	
	С тех пор я во всех ситуациях предпочитаю использовать редакторы, построенные 
	по принципу \, {\tt what~I~see~is~what~I~type}. Ну, и в этом жанре 
	{\tt Emacs} остается my all time favourite. С моей точки зрения, 
	все неявные форматы --- это ненужная и только мешающая надстройка. 
	Редакторы типа {\tt WordPerfect} или {\tt MSWord} пытаются совместить две
	различные и в значительной степени противоположные функции:
	собственно ввод текста с~логической разметкой, что и является функцией 
	текстового редактора, форматирование и вывод текста --- то, чем должны 
	заниматься typesetting systems. 
	\par
	Попытка совместить их в одном устройстве ни к чему хорошему не приводит,
	это все равно, что попытка совместить в одном устройстве фотоаппарат и 
	фотопринтер. Как известно, подобные многофункциональные устройства, 
	instant cameras, существуют (Polaroid, Kodak, Fujifilm и другие). Они удобны 
	в некоторых ситуациях, имеют собственную область применения и свой 
	рынок. Но только фотоаппаратами они от этого не~ста\-новятся. 
	
	%%%%%%%%%%%%%%%%%%%%%%%%%%%%%%%%%%%%%%%
	
	\section{MATHEMATICA}
	
	Однако параллельно нужно было учиться умножать матрицы. В августе 
	1990 года я посетил презентацию \, {\tt Mathematica 1.2} \,в Киото и
	чрезвычайно впечатлился, о чем и рассказал Лино, когда мы обсуждали
	наши планы относительно $(2,3)$-порождения. К этому моменту
	его миланский коллега Аурелио 
	Карбони\footnote{~\url{http://math.unipa.it/metere/Aurelio2013/index.html}}
	уже вовсю пользовался {\tt Mathematica}. 
	
	Ровно тогда в Италии
	происходила очередная экзацербация борьбы с нелицензионным 
	матобеспечением. Офицеры Guardia di Finanza {\it буквально\/}
	приходили в университет и проверяли установленные программы 
	на предмет легальности. Причем, вроде не только в общественных
	пространствах, но и в офисах профессоров. Так как наш проект
	финансировался M.U.R.S.T и C.N.R., мы с самого начала per forza
	были обречены пользоваться только лицензионными версиями.
	
	Если бы мы с Лино к тому моменту уже умели программировать,
	то, вероятно, выбрали бы в качестве основной системы \,{\tt Maple}, \,
	стиль программирования в котором значительно ближе к
	традиционным языкам программирования. Но, напомню, за все годы,
	которые меня учили <<программированию>>, мне не удалось написать
	ни одной работающей программы. 
	
	Поэтому мне не оставалось ничего другого, как взять второе издание
	книги Вольфрама \cite{W88} и прочесть ее в том же стиле, как я до
	этого читал Кнута, вслепую разбирая код. Степень моего невежества 
	на тот момент трудно описать. Я не то, что не знал, чем отличаются 
	компилируемые и интерпретируемые языки или процедурное и 
	функциональное программирование, я не знал программирования
	просто на уровне структур данных, уровней выражений, операций со 
	строками, разницы между Accuracy и Precision --- да чего там, между
	немедленным и отложенным присваиванием. Ретроспективно
	представляет интерес, каким образом мне удалось успешно сдать
	все эти 6--7 курсов по <<программированию>>? Действительно ли
	там не было всех этих слов или мне удалось настолько успешно их
	забыть? Может быть, и не было, кстати. Во всяком случае, в курсе
	классической механики за четыре семестра не было упомянуто то 
	{\it единственное\/} слово, которое нужно из него математикам, ---
	гамильтониан. 
	
	Поэтому, когда через неделю я смог написать первые {\it работающие\/}
	программы --- я~сейчас точно не помню, что это было, но, думаю, 
	что-то про полиномиальные матрицы~--- я {\it сильно удивился\/}. Когда
	через пару месяцев я смог сгенерировать системы корней для
	исключительных алгебр Ли и посчитать структурные константы в
	положительном базисе Шевалле (много позже эта работа была 
	опубликована \cite{V01}, на нее довольно активно ссылались
	физики), я удивился {\it еще больше\/}. В этот момент я понял, что, в 
	принципе, на бытовом уровне могу {\it сам\/} посчитать все, что понимаю
	как математик, и что все реальные ограничения, которые есть, если не 
	заниматься оптимизацией, связаны, главным образом, с {\it памятью\/} 
	компьютера (в меньшей степени с быстродействием).
	А~если нужно заниматься оптимизацией, то это все равно к доктору.
	
	За 3 месяца осени и зимы 1991/1992 я написал пару сотен страниц 
	вполне рабочего кода --- линейная алгебра, генерация матриц, 
	системы корней, символьные вычисления в алгебрах Ли, 
	алгебраических группах и представлениях, ну, и много еще разного.
	Мы с~Лино использовали его для экспериментов (= esplorazione) в 
	процессе работы над нашими статьями. Часть из этого потом
	дополнили и переработали мои ученики 
	Саша Лузгарев\footnote{~\url{http://www.mathnet.ru/php/person.phtml?option\_lang=rus\&personid=33774}}
	и Игорь Певзнер\footnote{~\url{https://atlas.herzen.spb.ru/teacher.php?id=3629}}, 
	совсем небольшая часть опубликована в \cite{VLP, VL}, потом
	Саша посчитал еще много чего, и разные люди этим вовсю пользовались.
	
	С тех пор я не умножал сам руками матрицы $3\times 3$, хотя в
	принципе умею это делать. И сейчас, пока я это пишу, тихо жужжит \,
	{\tt Mathematica 11.3}, \, хотя, вероятно, уже давно пора апгрейдить ее
	до следующей.
	
	Собственно, во введении к \cite{VHY} мы объясняем, почему 
	большинству математиков много легче выучить высокоуровневые языки 
	программирования, такие как \, {\tt Maple} \, или \,{\tt Mathematica}, \,в которых 
	2500--4000 слов, а не низкоуровневые, в которых 60--100 слов.
	С~точки зрения самой математики, мир низкоуровневого программирования
	соответствует стилистике математической логики с разбиением рассуждения
	на огромное число элементарных шагов. 
	
	
	В то же время подавляющее большинство математиков мыслит совершенно 
	иначе, большими блоками. Это то, о чем я пишу в \cite{V19}:
	``In the controversy of Physics vs Logic I stand with
	Physics. It is simply much larger. Specifically, in what concerns mathematical reasoning, 
	Logic attempts to break an argument
	into a huge number of elementary steps. The spirit of Mathematics is exactly the 
	opposite, to create most efficient ways of
	reasoning. Mathematical thinking consists of compressing huge bulks of arguments 
	into tangible entities that can be perceived
	as a whole by the human mind, and then to operate these pieces with very high 
	precision and certainty.''
	
	Иисус учил, что <<компьютер существует для человека, а не человек для 
	компьютера [пока]>>. Инструменты должны быть удобны тем, кто ими пользуется. 
	
	Математиков абсолютно необходимо учить программированию --- но именно как
	{\it алгоритмике\/}, в духе Кнута, с объяснением основных понятий и основных 
	идей, а вовсе не в духе пошагового кодирования. {\it Наоборот\/},
	результатом моего опыта явилось твердое убеждение, что основную массу
	пользователей вообще не нужно {\it ни в каком виде\/} учить тому, чему 
	в 1960-х, 1970-х и 1980-х годах безуспешно пытались учить меня, --- кодированию 
	в языках низкого уровня.
	
	%%%%%%%%%%%%%%%%%%%%%%%%%%%%%%%%%%%%%%%%%
	
	\section{FAST FORWARD}
	
	Осенью 1992 года я перебрался в Билефельд, на {\tt DEC} станцию (с номерами у 
	меня совсем плохо) под \,{\tt UNIX}. Переучивание происходило так. Ульф 
	Реманн\footnote{~\url{https://www.math.uni-bielefeld.de/\~rehmann/}} час или
	полтора порассказывал мне что-то с демонстрацией на своем компьютере ---  
	у него в офисе стояла такая же \,{\tt DEC}\, станция --- и вручил розовую картонку
	с макросами для \,{\tt Emacs}\, и мануал, страниц 40, мне кажется. Ну, макросы
	пришлось поучить дня три, все-таки их там штук 200, я уж сейчас не помню.
	
	И все чудесным образом заработало. Все файлы, которые я до этого 
	два года печатал в~трех разных странах под три другие платформы, прочитались
	и запустились без ошибок. В этот момент я понял, что \,{\tt UNIX}\, is very user friendly 
	(хотя, как известно, it is very selective about who its friends are).
	
	
	Книги Кнута \cite{K1, K2, K3} я прочел уже много позже, когда вернулся 
	в Петербург в конце 1990-х. И удивился, почему в студенческие годы мне никто не 
	объяснил, что это настоящая {\it серьезная\/} математика. Впрочем, мне
	кажется, я понимаю, почему. Потому что {\it тогда\/} никто из математиков не 
	понимал, что это математика. Ну, и уж, конечно, никто из программистов 
	не понимал тогда, а некоторые не понимают и сейчас, что это математика.  
	
	Потом я сам несколько лет преподавал курсы <<Прикладные математические пакеты>>, 
	<<Математика и компьютер>>, <<Алгоритмы и структуры данных>>, <<Криптография>>, 
	what not.
	Из этого выросли наши скрипты с Володей Халиным, Олегом Ивановым, Сашей
	Юрковым \cite{VILH, VHY}, и ссылки там. Ну, и много другого, о чем при случае. 		
		Вплоть буквально до того, что в прошлом году мы с Витей Петровым\footnote{~\url{ 
				https://math-cs.spbu.ru/people/petrov-v-a/}. Напомню, что в 2001 году Витя был 
			членом легендарной команды: Коля Дуров, Андрей Лопатин, Витя Петров, занявшей
			абсолютное первое место на чемпионате ACM в~Ванкувере, см.~\cite{AP}.} сняли 
		для Курсеры курс по
		{\it спортивному программированию\/} = competitive programming, вот
		продолжение курса \cite{KLST}, который Саша Куликов\footnote{~\url{  
				https://math-cs.spbu.ru/people/kulikov-a-s/}} снял с молодыми коллегами.
		Но, разумеется, в~нашем курсе, {\bf алгоритмы с числами}, все 
		придумывал Витя, и программы в псевдокоде писал он же \cite{PP}, а~я~выступал   
		в роли говорящей по-английски головы профессора.
		
		%%%%%%%%%%%%%%%%%%%%%%%%%%%%%%%%%%%%%%%%%
		
		\section{DIGITAL ASSISTANCE}
		
		В своей статье \cite{BB11} классики экспериментальной математики
		Дэвид Бэйли\footnote{~\url{https://www.davidhbailey.com/} ---  Computo ergo sum =
			я вычисляю, следовательно, я существую.}
		и Джонатан Борвайн\footnote{~\url{https://carma.newcastle.edu.au/resources/}. 
			Джонатан Борвайн скончался в 2016 году, но его коллеги в~CARMA поддерживают сайт
			со ссылками на ресурсы по экспериментальной математике.} 
		перечисляют следующие инструменты,
		использование которых радикально упростило для математиков
		поиск и генерацию нужной им информации. Ниже я просто воспроизвожу
		список из их статьи, понятно, что сегодня (да и тогда) к ним можно 
		было бы добавить многие дальнейшие аналогичные ресурсы:
		\par\smallskip\noindent
		{\tt $\backslash$quote}
		\par
		$\bullet$ {\bf Системы компьютерной алгебры} общего назначения,
		такие как {\tt Maple} и {\tt Mathematica} или даже \,{\tt Matlab}\footnote{~По этому
			поводу Володя Гердт и Коля Васильев оба прокомментировали, что \, {\tt Matlab}\, --- это
			пакет прикладных программ, а не полноценная система компьютерной алгебры, 
			а вот {\tt Axiom}, напротив, по замыслу самая интересная из всех таких систем. 
			И с тем и с другим я полностью согласен, но в данном случае просто
			воспроизвожу список Бэйли и Борвайна. Впрочем, и они тоже говорят ``or indeed 
			{\tt Matlab}''.} 
		и их бесплатные (= open-source) аналоги.
		\par\smallskip
		$\bullet$ {\bf Специализированные пакеты} такие, как {\tt CPLEX}, {\tt PARI},
		{\tt SnapPea}, {\tt Cinderella} и {\tt MAGMA}.
		\par\smallskip
		$\bullet$ {\bf Языки программирования} такие, как {\tt C}, {\tt C++} и 
		{\tt Fortran-2000}.
		\par\smallskip
		$\bullet$ Доступные в интернете {\bf математические приложения} такие, как 
		{\tt Sloane’s
			En\-cy\-clo\-pedia of Integer Sequences}, {\tt Inverse
			Symbolic Calculator}, {\tt Fractal Explo\-rer},
		{\tt Jeff Weeks’s Topological Games}, или {\tt Euclid}
		в {\tt Java}\footnote{~Ссылки на эти и некоторые дальнейшие ресурсы можно найти 
			на странице Бэйли \url{https://www.experimentalmath.info/}.}. 
		\par\smallskip
		$\bullet$ Базы данных и дальнейшие {\bf информационные ресурсы}
		{\tt Google}, {\tt MathSciNet}, {\tt arXiv}, {\tt Wikipedia}, {\tt MathWorld}, 
		{\tt MacTutor}, {\tt Amazon}, {\tt AmazonKindle} и многие другие,
		которые не всегда рассматриваются в таком качестве.
		\par\noindent
		{\tt $\backslash$unquote}
		\par\smallskip
		Опять же, I cannot agree more. С тем, что, конечно, в моем случае
		это были бы совершенно другие специализированные пакеты, {\tt GAP},
		{\tt CAYLEY}, {\tt Lie}, {\tt Chevie}, {\tt Singular}, {\tt CoCoA},
		{\tt Fermat}, {\tt Macaulay}, \ldots, what not. Ну, и куда делась
		{\tt Axiom}, в конце концов? С тем, что если бы я начинал заниматься 
		этим не 30 лет назад, а сегодня, то это, разумеется, был бы {\tt Sage}. 
		А так все верно, разумеется.
		
		Что касается информационных ресурсов, то, кроме непосредственно
		баз данных,  это тысячи сайтов индивидуальных математиков, 
		семинаров, университетов, институтов, математических обществ, библиотек, 
		журналов, издательств, на которых почти всегда удается за пару минут 
		найти препринт или электронную копию почти
		любой опубликованной статьи или книги. Такая возможность принципиально
		изменила доступность научной информации. Но это, конечно, неспецифично 
		для математики. 
		
		Ну, и, конечно, не только для математики, но для математики в первую очередь 
		я бы назвал еще два типа продуктов,
		которые за последние 30 лет полностью изменили облик математических текстов.
		\par\smallskip
		$\bullet$ {\bf Системы компьютерного набора}, системы кодировки шрифтов 
		и символов, текстовые редакторы, в первую очередь {\tt \TeX},  его диалекты, 
		расширения и оболочки, но и, например, {\tt Unicode}.
		\par
		В отношении текстовых редакторов
		было бы трудно сказать, что оказалось столь же полезным для математиков.
		В моем случае, очевидно, {\tt Emacs}, а также российские редакторы ASCII-only и заточенные
		под \,{\tt\TeX}\, системы типа \,{\tt WinEdt} \,или \,{tt {\TeX}works}, \,но  
		я понимаю, насколько мой опыт нерепрезентативен.
		\par\smallskip
		$\bullet$ Языки компьютерной графики, в первую очередь {\tt PostScript}, 
		графические редакторы и {\bf системы компьютерной графики}. 
		\par
		Тут мой опыт еще менее репрезентативен. В Билефельде мы все рисовали в
		\,{\tt Xfig}\footnote{~В нашей статье \cite{PSV} с Женей Плоткиным и Андреем
			Семеновым нет ни одной теоремы. Зато есть одна лемма и несколько десятков
			картинок. Все их запечатлел в {\tt Xfig} Андрей по нашим с Женей зарисовкам
			от руки. И в самом деле, какая польза от математических статей без картинок?}. 
		Когда я только
		вернулся в Россию, Иосиф Владимирович Романовский\footnote{~\url{
				http://kio-math.spbu.ru/kio\_roma.html}} рассказал мне, что он сам
		программирует картинки в своих текстах в \,{\tt PostScript}\, непосредственно
		в~координатах. Я, конечно, на такие подвиги не способен и после нескольких
		несистематических попыток использования  {\tt Adobe Illustrator} и 
		\,{\tt CorelDraw}\, и рисования в \,{\tt\TeX}\, перешел к заданию картинок
		системами уравнений и неравенств в {\tt Mathematica}. Работать 
		в этой технике легче, чем кажется. 
		Например, преподавая студентам компьютерную графику, я и сам
		научился изображать традиционные японские паттерны (seikaiha и пр.),
		орнаменты на плоскости, представляющие плоские кристаллографические 
		группы (wallpaper groups), etc. Но я ленюсь.
		\par\smallskip
		Впрочем, любой подобный список неминуемо не полон. Например, можно спросить,
		а где же все языки и {\bf системы формального вывода}, theorem provers и все такое?
		Где {\tt Theorem}, {\tt Coq}, {\tt Isabelle}, {\tt Mizar}, {\tt HOL light},\ldots? Видимо, 
		для Бэйли и Борвайна, как и для меня, это не то, что реально повлияло на 
		восприятие классической математики. В любом случае, пока компьютерная
		математика {\it в этом смысле\/} --- это не математика, а некая совершенно 
		отдельная деятельность. Настолько же отдельная, как изготовление, настройка
		и ремонт музыкальных инструментов по отношению к музыке.
		Я с уважением отношусь ко многому из того, что там происходит, но не готов
		подписаться под очередной проект реорганизации всей математики в духе
		Бурбаки, только с выбрасыванием на этот раз не 9/10, а 99/100. И обсуждаю 
		здесь нечто {\it абсолютно\/} другое, не имеющее никакого отношения к 
		формальному выводу.
		
		%% http://www.cse.chalmers.se/research/group/logic/TypesSS05/Extra/wiedijk_2.pdf
		
		%%%%%%%%%%%%%%%%%%%%%%%%%%%%%%%%%%%%%%%%%
		
		\section{НОВАЯ РЕАЛЬНОСТЬ}
		
		Перечислю с совсем короткими комментариями несколько абсолютно фундаментальных 
		изменений, произошедших за последние десятилетия. Здесь нет, конечно,
		возможности хоть сколько-нибудь развить эти темы. Некоторые из них обсуждаются 
		в~нашей книге  \cite{VHY} с Володей Халиным\footnote{~\url{
				http://www.econ.spbu.ru/ru/people/halin-vladimir-georgievich}}
		и Сашей Юрковым\footnote{~\url{ 
				http://www.econ.spbu.ru/ru/people/yurkov-aleksandr-vasilevich}}.
		К некоторым другим~я~намереваюсь подробнее вернуться в следующих частях.
		
		\par\smallskip
		$\bullet$ {\bf Контакт с реальностью.} Мы получили возможность непосредственно
		наблюдать математическую реальность в гораздо большем объеме, чем это
		было доступно когда-либо ранее.
		Джаффе и Квинн говорят: ``Mathematics may have even better experimental 
		access to mathematical reality than the laboratory sciences have to physical reality,''
		--- я полностью согласен с этим, естественно, с заменой смысла слов 
		<<теоретическая математика>> и <<экспериментальная математика>> на прямо 
		противоположный, по сравнению со статьей \cite{JQ}. Для меня формальное
		доказательство математической теоремы является таким же сырым экспериментальным
		материалом, как компьютерное вычисление само по себе. 
		
		\par\smallskip
		$\bullet$  {\bf Соотношение идей и вычислений.} Начиная с XIX века в математике 
		происходила борьба двух установок
		\par\smallskip
		--- {\bf заменить идеи вычислениями} (фон Лейбниц),
		\par\smallskip
		--- {\bf заменить вычисления идеями} (Дирихле).
		\par\smallskip\noindent
		В XX веке она привела к реальным эксцессам: <<К бессвязным наборам 
		бессмысленных и бессодержательных калькулятивных эксерсисов с одной
		стороны и артефактам для артефактов с другой>> \cite{VHY};
		``Ce qui limite le vrai, ce n'est pas le faux, c'est l'insignifiant''~\cite{Ch}.
		\par
		Теперь мы получили гораздо большую возможность, чем когда-либо имели
		математики после XVIII века, {\bf соединить идеи с вычислениями}.
		<<Настоящая математика — как и настоящее программирование — основаны на игре 
		и равновесии идей и вычислений. Любую идею можно превратить в вычисление, и 
		любое правильно организованное вычисление может привести не только к результату, 
		но и к пониманию>> \cite{VHY}.
		
		\par\smallskip
		$\bullet$  {\bf Конечное и бесконечное.} 
		Пожалуй, один из главных философских уроков, который сразу извлекает каждый, 
		кто пытался произвести какое-то компьютерное вычисление, состоит в том, что 
		между конечным и бесконечным нет никакой разницы с точки зрения их
		{\it практической осуществимости\/}. В некотором cовершенно точном смысле 
		иерархия высших бесконечностей моделируется внутри натурального ряда, а
		сама возможность вычислений с {\it большими\/} конечными числами 
		предполагает существование более высоких бесконечностей, чем те, которые 
		живут в конструктивном геделевском универсуме $V=L$.
		\par
		В 1955 году Давид ван Данциг спрашивал, является ли  
		$10\uparrow\uparrow 3=10^{\displaystyle10^{10}}$ конечным числом \cite{D}.
		Спустя 20 лет для Кнута \cite{K76} моделью большого числа служит, скорее, 
		что-то вроде $10\uparrow\uparrow\uparrow\uparrow 3$. Если с тех пор эта граница
		{\it психологически\/} еще немного отодвинулась, то не так сильно: ``Advances 
		in our ability to compute are bringing us substantially closer to ultimate limitations''
		\cite{K76}.
		\par\smallskip
		$\bullet$ {\bf Возможное и невозможное.} В действительности подлинный
		Вызов --- это не унаследованная нами от классической древности
		{\it фиктивная\/} альтернатива конечного и бесконечного, а реальная альтернатива
		{\bf маленького и большого\/}, возможного и невозможного. Маленькие
		бесконечности $N$, $\omega$ и $\aleph_0$ столь же доступны нашей интуиции,
		как число~1. Но разница между
		$10\uparrow\uparrow\uparrow\uparrow\uparrow\uparrow 3$ и
		$10\uparrow\uparrow\uparrow\uparrow\uparrow\uparrow\uparrow 3$
		значительно превосходит наше воображение.
		\par
		Продолжу цитату из Кнута:
		``Finite numbers can be really enormous, and the known universe is very small.
		Therefore, the distinction between finite and infinite is not as relevant as the
		distinction between realistic and unrealistic'' \cite{K76}. Именно так, с тем,
		что сегодня мы,
		пожалуй, предпочли бы говорить об {\bf осуществимом и неосуществимом}
		= feasible and unfeasible.
		\par\smallskip
		$\bullet$ {\bf Проблема промежуточных размеров.} При решении многих 
		математических задач снова и снова возникает одна и та же ситуация. Для достаточно больших порядков, степеней, размерностей, \ldots \ удается провести
		общее доказательство. Для маленьких все легко просчитать явно, руками 
		или на компьютере. Но между маленьким~и~большим остается более или
		менее обширная зона объектов промежуточных размеров~= {\bf intermediate size}, 
		которые слишком велики для явного перебора, но слишком малы, 
		чтобы к ним была применима общая теория.
		\par
		Вот как, например, совсем до недавнего времени эта зона выглядела в
		аддитивной теории чисел:
		``Nous constatons d\`es lors qu’il reste une zone extr\^emement \'etendue,
		typiquement entre $10^{10}$ et $10^{100 000}$ o\`u les moyens de calcul standarts ne
		suffisent plus et o\`u les m\'ethodes analytiques asymptotiques sont encore
		inop\'erantes'', \cite{R}.
		\par
		Конкретное местоположение и величина этой серой зоны могут меняться от 
		задачи к задаче, но само это явление настолько повсеместно, что заслуживает,
		как мне кажется, серьезного специального анализа. Классическая математика прекрасно 
		работала в~двух предельных случаях. <<Дискретная>> математика в случае явно 
		решаемых моделей и <<непрерывная>> математика в случае, когда число
		рассматриваемых объектов настолько велико, что может считаться бесконечным.
		
		
		Не исключено, что ограниченный успех применения математики в таких
		областях, как биология, психология, анализ языка, исторических, социальных 
		и экономических 
		явлений и т.\,д. связан --- кроме отсутствия вменяемых моделей в самих этих
		предметных областях --- именно с тем, что они попадают в такую серую
		зону. Слишком велики для перебора и слишком малы для общей теории.
		
		Разумеется, {\it номинально\/} все современные компьютерные науки о данных 
		(= data science), искусственный интеллект (= artificial intelligence), нейронные
		сети (= neural networks) и пр. занимаются
		именно этим, но никаких содержательных %% математических 
		идей, кроме того, чтобы увеличить используемый корпус текстов или фотографий
		еще в $10^6$ раз \cite{HNP}, там, вроде бы, пока нет. Конечно, довольно 
		часто уже просто увеличение используемой базы данных в $10^6$ раз приводит 
		к совершенно поразительным практическим результатам.
		
		\par\smallskip
		$\bullet$ {\bf Детерминированное и случайное.} Другим важнейшим 
		философским уроком последних десятилетий является снятие еще
		одной {\it фиктивной\/} альтернативы прошлого, а~именно, 
		противопоставления детерминированного и случайного.
		
		Я здесь не говорю о вероятностных {\it алгоритмах\/}, которые работают в
		миллиарды раз быстрее, чем детерминированные и при этом дают абсолютно
		достоверные (а не вероятностные) ответы. Не говорю о вероятностных 
		{\it результатах\/}, утверждающих, что два произвольных элемента порождают 
		данную группу (с вероятностью 1), но при этом ни одной такой пары
		фактически не известно. Или о различении массового --- простого ---
		случая и трудных исключений, которым приходится посвящять 99\,\% 
		анализа. Не говорю и о трудностях предсказания погоды в Вене на 
		Рождество 2050 года. Ко всему этому я~планирую вернуться в следующих 
		частях этого текста.
		
		Здесь я имею в виду то, что у человека нет никаких надежных априорных 
		механизмов, чтобы отличать подлинно случайное от высоко организованного.
		В действительности, как видно из многих поразительных примеров, таких как теорема о полярном круге (Arctic circle theorem~\cite{JPS}),
		подлинно случайное чрезвычайно регулярно. А вот то, что {\it кажется\/} 
		нам случайным, на самом деле представляет собой редчайшее 
		явление и~должно быть специально сконструировано.
		
		
		\par\smallskip
		$\bullet$ {\bf Выводимое и наблюдаемое.} Еще одна альтернатива
		из прошлого, которая, как оказалось, лишена какого-либо реального содержания, ---
		это противопоставление аналитического и 
		синтетического знания. Все следствия содержатся в аксиомах, поэтому 
		математики занимаются выводом тавтологий, а тавтологии не говорят 
		ничего (``Tautologie und Kontradiktion sind sinnlos'').
		
		
		Ну да, there are more things in Heaven and Earth, Horatio, than
		man was supposed to know. 
		То, о  чем философы не подозревали, --- это {\bf непредсказуемая сложность}
		вывода следствий из простых аксиом, то, что Стивен Вольфрам в своей 
		книге \cite{W02} называет computational irreducibility. Можно по-разному
		относиться к его идеям, но то, что две-три строки кода могут после
		нескольких итераций породить результат совершенно непредсказуемой 
		сложности --- это просто факт жизни, который нужно принять: ``But most of 
		what’s powerful out there in the computational universe is rife with computational 
		irreducibility — so the only real way to see what it does is just to run it and watch 
		what happens'', \cite{W02}. ``Even when the underlying rules for a system are 
		extremely simple, the behavior of the system as a whole can be essentially 
		arbitrarily rich and complex'' \cite{W20}.
		
		Сюда относятся, в частности, фрактали\footnote{~Я догадываюсь, что по-русски 
			слово ``фрактал''
			чаще используют в мужском роде. Для меня это звучит примерно, как ``горизонтал'',
			``вертикал'' и ``диагонал''. Исходно по-французски ``une fractale'' в женском
			роде. Но, удивительнейшим образом, по-итальянски ``un frattale'', хотя форма слова 
			подразумевает обе возможности, и ``una frattale'' казалось бы мне {\it гораздо\/} 
			более естественным, по аналогии с ``una verticale'', ``una diagonale''. По-испански
			и по-португальски тоже ``un fractal'' и ``um fractal'', соответственно. Но по-каталански 
			все же правильно ``una fractal''.}, странные аттракторы, классический хаос, and 
		all that. Тот, кто считает, что вывод следствий из аксиом не порождает нового знания,
		может взять какую-нибудь простейшую {\it классическую\/} динамическую систему,
		вот хотя бы двойной маятник, и попробовать посредством боевого
		омфалоскепсиса и феноменологической интроспекции предсказать ее траекторию
		при заданных начальных условиях. Can you do that? Really? А ведь большинство 
		математических теорий и других окружающих нас вещей устроены чуть сложнее, чем 
		двойной маятник.
		
		В свете этого утверждение, что вся невообразимая сложность окружающего нас мира
		может быть сгенерирована одной--двумя страницами кода, не представляется уже 
		столь фантастичным.
		%% barbaro osservador, ma diligente.
		\par\smallskip
		Следующие три аспекта носят скорее не философский, а педагогический и 
		социологический характер, но тоже представляются мне довольно важными
		в практическом плане.
		\par\smallskip
		$\bullet$ {\bf Алгоритмическое мышление.}
		<<В большинстве случаев принятое сегодня изложение математики — не
		только в школе, но и на математических факультетах университетов — 
		возникло в докомпьютерную эпоху и совершенно неудовлетворительно 
		с алгоритмической точки зрения. Между тем, часто даже небольшое
		изменений определений делает их значительно более пригодными для 
		практических вычислений. Скажем, не только в школьном, но и в 
		университетском курсе, степень определяется рекурсивно как $x^n=x^{n-1}x$. 
		Между тем,
		самое незначительное изменение этого определения может драматически
		увеличить его применимость. Например, можно определить степень 
		указанной выше формулой в случае, когда $ n $ нечетно, и формулой 
		$x^n={(x^{n/2})}^2$ в случае, когда $n$ четно. Для объектов, умножение 
		которых занимает заметное время (матрицы или многочлены высокой степени), 
		вычисление,
		скажем, 1000-й степени с использованием второго определения может быть
		в сотни раз быстрее, чем с помощью первого. Заметим, что речь не идет о
		наиболее эффективном профессиональном алгоритме для вычисления 
		степени — в большинстве случаев достаточно просто задумываться над 
		подобного рода нюансами>> \cite{VHY}.
		
		Это именно то, что мы стараемся делать сейчас на Факультете Математики и 
		Компьютерных Наук\footnote{~\url{https://math-cs.spbu.ru/}}, учить настоящей 
		математике, как мы ее понимаем, но при этом так, чтобы она включала в себя 
		Computer Science: алгоритмы, сложность и все такое.
		
		\par\smallskip
		$\bullet$ {\bf Связь времен.} 
		Достаточно просто бегло пролистать два классических учебника алгебры,
		``Lehrbuch der Algebra'' Генриха Вебера 
		\cite{W}\footnote{~Два первых тома стоят в Лаборатории алгебры и теории чисел ПОМИ, 
			комната 306. Третий том
			``Elliptische Funktionen und algebraische Zahlen'' более специальный, был 
			напечатан отдельно, и включать его в сравнение с учебником ван дер Вардена 
			в любом случае не имеет смысла.} 
		и ``Moderne Algebra'' Бартеля ван дер Вардена \cite{Wa},
		чтобы оценить {\it ошеломляющий\/} масштаб той гильбертовской {\bf революции в~математике}, 
		которая произошла между 1896 и 1926 годами \cite{Q}. В книге Вебера каждая 
		страница насыщена конкретными вычислениями и разбором примеров. В
		книге ван дер Вардена вычисления сведены к абсолютному минимуму и всюду,
		где возможно, заменяются общими рассуждениями.
		
		Я сам формировался как математик в эпоху Бурбаки, когда любое доказательство
		длиннее, чем в три строчки, считалось плохо организованным и не до конца 
		продуманным. Его следовало продумать и разбить на последовательность 
		очевидных лемм. По крайней мере, именно таков был господствующий Zeitgeist.
		Читать с таким настроением классиков, вот хоть того же Эйлера, было довольно
		трудно. Что уж говорить о формулах, которые писал Рамануджан. Они вызывали
		изумление, часто восхищение, но порой и~раздражение. 
		
		Системы компьютерной алгебры позволили заново переподключиться ко {\it всей\/} 
		истории математики, без лакун и изъятий. Ну, понятно, что самому Рамануджану 
		формулы диктовала богиня 
		Намаккал. Но сегодня системы компьютерной алгебры с огромным успехом пишут 
		формулы ровно в том же стиле, как Эйлер и Рамануджан. Просто \,{\tt Mathematica}\,
		и \,\,{\tt Maple}\,\, знают
		массу вещей (специальные функции, гипергеометриче\-ские ряды, what 
		not), которые, в силу своей специфики, не умещались в обучение математиков
		моего поколения, и не боятся считать чуть более глубоко и упорно, чем это
		привыкли делать мы.
		
		\par\smallskip
		$\bullet$  {\bf Теоретическая и экспериментальная математика.} С предыдущим
		пунктом теснейшим образом связан и другой, абсолютно принципиальный момент, ---
		возможность участия широких кругов образованных людей, студентов и 
		школьников в {\it серьезной\/} исследовательской работе. Здесь под <<исследованиями>>
		я, разумеется, имею в виду не тот смысл, который это слово получило сегодня,
		research,
		а исходный его смысл, {\bf exploration}.
		
		Я не призываю к разделению {\it самой\/} математики на теоретическую и экспериментальную
		(хотя движение в эту сторону в любом случае происходит, совершенно независимо 
		от того, призываем мы к этому или нет \cite{BB05, BB11}). Но компьютерная 
		математика дает возможность миллионам людей наблюдать математический мир 
		в том же смысле и с~теми же устремлениями, как миллионы людей смотрят на 
		небо через любительские телескопы. А в том, чтобы резко расширился круг людей,
		способных получать от математики эстетическое и {\it когнитивное\/} удовольствие
		того же рода, как от музыки или живописи (не будучи при этом сами 
		профессиональными музыкантами или художниками), я~как раз не видел бы 
		{\it никакого\/} вреда.
		
		%%%%%%%%%%%%%%%%%%%%%%%%%%%%%%%%%%%%%%%%%%
		
		\section{ЗАКЛЮЧЕНИЕ}
		
		
		Этот опыт радикально изменил мое понимание того, что такое математика, что умели и 
		что не умели считать наши предшественники, что вообще можно и что нельзя
		посчитать {\it любыми\/} средствами, что можно и что нельзя посчитать, пользуясь 
		{\it сегодняшними\/} профессиональными инструментами, что лично я могу и не могу 
		посчитать и сколько примерно времени и места это все может занять, как написание 
		программы, так и~само вычисление.
		
		Все наши математические курсы всех уровней должны быть радикально 
		перестроены с учетом возможностей сегодняшних --- и завтрашних --- систем 
		компьютерной алгебры. Если мы сможем достойно ответить на связанные с
		этим социальные и педагогические вызовы, то я не вижу в этом никаких угроз
		для математики, только возможности. Воплощая эстетику {\it исторического\/} 
		оптимизма, я вынужден констатировать, что прошлое математики 
		удивительно, а ее настоящее прекрасно. И то  будущее, которое выше всяких 
		представлений, тоже неизбежно наступит. Но вот когда и где, зависит от того, 
		как математические и образовательные сообщества в целом сумеют справиться 
		со своей задачей очистить и передать.
		
		
		\subsection{Посвящение} Абсолютно ключевой человек, с которым мы все это 
		постоянно обсуждали последние 20 лет, и который должен был
		бы фигурировать в благодарностях, это Олег Иванов. С~Олегом мы 
		познакомились в школе, нас связывало 50 лет дружбы, все это время мы учились 
		друг у друга, и я ему многим обязан. Последние годы он много экспериментировал 
		с системами компьютерной алгебры в преподавании, и на школьном \cite{I1, I2, I3,
			I4, I5, I6} и
		на университетском уровне \cite{VILH, Iv, IF}.
		Он умел объяснять понятно, сохраняя при 
		этом интригу, что редко кому удается. Его книга \cite{I} --- это абсолютная классика 
		в объяснении духа математики людям любой подготовки, вот такая же, как книги 
		Клейна \cite{Kl} или Пойи~\cite{P}. Одна из очень немногих книг, которые, дочитав 
		первый раз до конца, я тут же снова начал читать сначала.
		
		\subsection{Благодарности}
		Я благодарен Андрею Семенову, Володе Халину и Саше Юркову за многолетнее 
		сотрудничество в продвижении компьютерной алгебры в преподавание как
		математикам, так и нематематикам.
		Все эти годы исключительно важным для меня было постоянное общение с 
		настоящими профессионалами в области компьютерной математики Володей 
		Гердтом и Колей Васильевым. С Андреем Родиным мы очень интенсивно 
		обсуждаем принципиальные аспекты всего, о чем здесь рассказывается.
		Наконец, совершенно особая благодарность Сергею Позднякову, который убедил 
		меня написать этот цикл статей, за настойчивость и многочисленные 
		стимулирующие обсуждения. Я признателен всем им за интересные и глубокие 
		комментарии к первому варианту этой статьи, исправления и ссылки.
		
		%%%%%%%%%%%%%%%%%%%%%%%%%%%%%%%%%%%%%%%%


\subsection{Disclaimer}
Высказанные здесь взгляды и оценки являются исключительно моими собственными
и не выражают официальную позицию никакого учреждения или профессиональной
организации, с которыми я аффилиирован.
%%  sine ira cum studio

%%%%%%%%%%%%%%%%%%%%%%%%%%%%%%%%%%%%%%%%%%

\begin{thebibliography}{99}
{\footnotesize 	
	\bibitem{AP} {\it Асанов М.~О., Парфенов В.~Г.} Финальные соревнования чемпионата 
	мира по программированию потрясающий успех петербургских команд // Компьютерные
инструменты в образовании. 2001. №~2. С. 11--18.
	
	\bibitem{VL} {\it  Вавилов Н.~А., Лузгарев А.~Ю.} Группа Шевалле типа 
	$\E_7$ в 56-мерном представлении // Зап. научн. сем. ПОМИ. 2011. Т. 386. C. 5--99.
	
	
	\bibitem{VLP} {\it  Вавилов Н.~А., Лузгарев А.~Ю., Певзнер И. М.}
	Группа Шевалле типа $\E_6$ в 27-мерном представлении // Зап. научн. сем. ПОМИ. 2006.
	Т. 338. C. 5--68. 
	
	\bibitem{VILH} {\it Вавилов Н.~А., Иванов О.~А., Лушникова Г.~А., Халин В.~Г.}
	Уроки математики при помощи {\tt Mathematica}. СПб.: ОЦЭиМ,  2008. 146 с.
	
	\bibitem{VHY} {\it Вавилов Н.~А., Халин В.~Г., Юрков А.~В.}
	{\tt Mathematica} для нематематика. СПб., 2020. 484 с.
	
	\bibitem{I} {\it Иванов О.~А.\/} Избранные главы элементарной математики. СПб.: Изд-во
	СПбГУ, 1995. 223 c.
	
	\bibitem{I1} {\it Иванов О.~А.\/} {\tt Maxima} в обучении математике в школе.
	Урок 1. Введение в предмет //
	Компьютерные инструменты в школе. 2010. №~1. С. 5--14.
	
	\bibitem{I2} {\it Иванов О.~А.\/} {\tt Maxima} в обучении математике в школе.
	Урок 2. Компьютерные эксперименты с {\tt Maxima} //
	Компьютерные инструменты в школе. 2010. №~2. С.~3--13.
	
	\bibitem{I3} {\it Иванов О.~А.\/} {\tt Maxima} в обучении математике в школе.
	Урок 3. Элементы программирования в {\tt Maxima} //
	Компьютерные инструменты в школе. 2010. №~3. С.~3--12.
	
	\bibitem{I4} {\it Иванов О.~А.\/} {\tt Maxima} в обучении математике в школе.
	Урок 4. Алгебра с {\tt Maxima} //
	Компьютерные инструменты в школе. 2010. №~4. С.~3--13.
	
	\bibitem{I5} {\it Иванов О.~А.\/} {\tt Maxima} в обучении математике в школе.
	Урок 5. Математический анализ с {\tt Maxima} //
	Компьютерные инструменты в школе. 2010. №~5. С.~3--15.
	
	\bibitem{I6} {\it Иванов О.~А.\/} {\tt Maxima} в обучении математике в школе.
	Урок 6. Заключение: идеи, принципы, подходы // 
	Компьютерные инструменты в школе. 2010. №~6. С.~3--13.
	
	\bibitem{Iv} {\it Иванов О.~А.\/}
	Элементарная математика для школьников, студентов и преподавателей.
	М.: МЦНМО,   2019. 390 с.
	
	\bibitem{IF} {\it Иванов О.~А., Фридман Г.~М.\/} Дискретная математика
	и программирование в {\tt Wolfram Mathematica}. СПб.: Питер,  2019. 352 с.
	
	\bibitem{Kl} {\it Клейн К.~Ф.}
	Элементарная математика с высшей точки зрения. М.: Наука. Т. 1. 
	Арифметика, алгебра, анализ. 1987. 431 с.; Т. 2. Геометрия. 1987. 416 с.
	
	\bibitem{K} {\it Кнут Д.~Э.\/} Все про {\tt \TeX} = The {\tt \TeX} Book,
	%%% (Перевод с англ. М.~В.~Лисиной), 
	Протвино: АО RDTeX, 1993. 592 c. 
	
	\bibitem{K1} {\it Кнут Д.~Э.\/} Искусство программирования. I. Основные алгоритмы. 
	М.--СПб.--Киев: Вильямс,  2000. 712 с.
	
	\bibitem{K2} {\it Кнут Д.~Э.\/} Искусство программирования. II. Получисленные алгоритмы. 
	М.--СПб.--Киев: Вильямс, 2000. 827 с.
	
	\bibitem{K3} {\it Кнут Д.~Э.\/} Искусство программирования. III. Сортировка и поиск.
	М.--СПб.--Киев: Вильямс, 2000. 822 с.
	
	\bibitem{M1} {\it Манин Ю.~И.\/} Доказуемое и недоказуемое. М.: Советское радио, 1979. 89 с.
	
	\bibitem{M2} {\it Манин Ю.~И.\/} Вычислимое и невычислимое. М.: Советское радио, 1980. 128 с.
	
	\bibitem{PP} {\it Петров В.~А, Поздняков С.~Н.} Заочная школа современного 
	программирования. Занятие 1. алгоритмы над целыми числами //  Компьютерные
	инструменты в образовании. 1999. №~1. С. 39--49.
	
	\bibitem{P} {\it Пойа Дж.} Математика и правдоподобные рассуждения.
	%% под редакцией С.А.Яновской. Пер. с английского И.А.Вайнштейна.
	М.: Наука, 1975. 464 с.
	
	\bibitem{Se} {\it Семенов А.~А.\/} Разложение Брюа длинных корневых торов
	в группах Шевалле: канд. дисс. СПб.: СПбГУ,  1991. С. 1--143.
	
	\bibitem{Sp} {\it Спивак М.~Д.\/} Восхитительный {\tt \TeX}: руководство по
	комфортному изготовлению научных публикаций в~пакете  {\tt AMS-\TeX} =
	The joy of {\tt \TeX}: A gourmet guide to typesetting with the {\tt AMS-\TeX} 
	macro package. М.: Мир, 1993. 288 с.
	
	\bibitem{St1} {\it Штейнгауз Г. \/} Задачи и размышления. М.: Мир, 1974.
	
	\bibitem{BB05} {\it  Bailey, D.~H., Borwein J.~M.\/} Experimental 
	mathematics: examples, methods and implications // 
	Notices Amer. Math. Soc. 2005. Vol.~52, №~5. P. 502--514. 
	
	\bibitem{BB11} {\it  Bailey, D.~H., Borwein J.~M.\/} Exploratory experimentation 
	and computation // Notices Amer. Math. Soc. 2011. Vol.~58, №~10. P. 1410--1419. 
	
	\bibitem{C} {\it Carlin G.\/} Brain droppings. Hyperion, 1998. 272 p.
	
	\bibitem{Ch} {\it Chenciner A.\/} Le vrai, le faux, l’insignifiant.
	Mat\'eriaux pour une discussion de deux phrases de Ren\'e Thom.
	Texte \'ecrit à l'occasion d'un expos\'e au colloque Phenomath (22 mai 2015). P. 1--12.
	URL: \url{https://perso.imcce.fr/alain-chenciner/Vrai\_faux\_insignifiant.pdf} (дата обращения: 22.06.2020).
	
	\bibitem{CCNPW} {\it Conway J.~H., Curtis R.~T., Norton S.~P., Parker R.~A., 
		Wilson R.~A.\/} Atlas of finite groups. Maximal subgroups and ordinary characters 
	for simple groups. With computational assistance from J. G. Thackray. NY: Oxford University Press, Eynsham, 1985. 252 p.
	
	\bibitem{D} {\it van Dantzig D.\/} Is $10^{10^{10}}$ a finite number? //
	Dialectica. 1955. Vol. 9. P. 273--277. 
	
	\bibitem{DT} {\it Di Martino L., Tamburini, M.~C.\/} $2$-generation of finite simple groups 
	and some related topics // Generators and relations in groups and geometries (Lucca, 1990). 
	NATO Adv. Sci. Inst. Ser. C Math. Phys. Sci. Kluwer Acad. Publ., 
	Dordrecht. 1991. Vol. 333. P. 195--233.
	
	\bibitem{HNP} {\it Halevy A., Norvig P., Pereira F.} The unreasonable effectiveness of data //
	Intelligent Systems. IEEE. 2009. Vol. 24, №~2. P. 8--12.
	
	\bibitem{H1} {\it Helfgott H.~A.\/} Minor arcs for Goldbach's problem //
	arXiv:1205.5252v4  [math.NT]  30 Dec 2013. P. 1--79.
	
	\bibitem{H2} {\it Helfgott H.~A.\/} Major arcs for Goldbach’s problem //
	arXiv:1305.2897v4 [math.NT] 14 April 2014. P. 1--79.
	
	\bibitem{H3} {\it Helfgott H.~A.\/} The Ternary Goldbach Conjecture is true //
	arXiv:1312.7748v2  [math.NT]  17 Jan 2014. P. 1--79.
	
	\bibitem{H4} {\it Helfgott H.~A.\/} The ternary Goldbach problem // 
	arXiv:1501.05438v2  [math.NT]  27 Jan 2015. P. 1--327.
	
	\bibitem{HP} {\it Helfgott H.~A., Platt D.~J.\/} Numerical verification of the ternary 
	Goldbach conjecture up to $8.875\cdot 10^{30}$ //
	Exp. Math. 2013. Vol. 22, №~4. P. 406--409.
	
	\bibitem{JQ} {\it Jaffe A., Quinn F.\/} ``Theoretical mathematics'': toward a cultural 
	synthesis of mathematics and theoretical physics~// Bull. Amer. Math. Soc. (N.S.) 1993.
	Vol. 29, №~1. P. 1--13.
	
	\bibitem{JPS} {\it Jockusch W., Propp J., Shor P.}
	Random domino tilings and the arctic circle theorem // 
	arXiv:math/9801068v1 [math.CO], 13 Jan 1998. P. 1--46. 
	
	\bibitem{K76} {\it Knuth D.~E.\/} Mathematics and computer science: coping with 
	finiteness // Science. 1976. Vol. 194, №~4271. P.~1235--1242.
	
	\bibitem{KLST} {\it Kulikov A.~S., Logunov A., Simonov K., Tolstikov A.\/}
	Competitive programmer's core skills // Coursera. [Online Course]
	\url{https://www.coursera.org/learn/competitive-programming-core-skills} (дата обращения: 22.06.2020).
	
	\bibitem{PSV} {\it Plotkin E., Semenov A., Vavilov N.\/} Visual basic representations: 
	an atlas // Internat. J. Algebra Comput. 1998. Vol.~8, №~1. P. 61--95.
	
	\bibitem{Q} {\it Quinn F.} A revolution in mathematics? What really happened 
	a century ago and why it matters today // Notices Amer. Math. Soc. 2012. Vol. 59, 
	№ 1. P. 31--37. doi: 10.1090/noti787
	
	\bibitem{R} {\it Ramar\'e O.\/} \'Etat des lieux. [Online]. URL:
	\url{http://iml.univ-mrs.fr/~ramare/Maths/ExplicitJNTB.pdf}. P. 1--19 (дата обращения: 22.06.2020).
	
	%% \bibitem{Se} {\it Sejnowski T.~J.} The unreasonable effectiveness of deep
	%% learning in artificial intelligence, {\tt www.pnas.org/cgi/doi/10.1073/pnas.1907373117}, 
	%% 2020, 1--6.
	
	\bibitem{St} {\it Steinhaus H.\/} Mi\c{e}dzy duchem a materi\c{a} po\'sreniczy 
	matematyka. PWN: Warszawa--Wroc\l{}aw, 2000. 261 p.
	
	\bibitem{T} {\it Tamburini M.~C.\/} Generation of certain simple groups by elements 
	of small order // Istit. Lombardo Accad. Sci. Lett. Rend. A. 1987. Vol. 121. P. 21--27.
	
\bibitem{V90} {\it Vavilov N.\/} Structure of Chevalley groups over commutative rings //
Nonassociative algebras and related topics (Hiroshima, 1990), World Sci. Publ., 
River Edge, NJ, 1991. P. 219--335. 
	
	\bibitem{V01} {\it Vavilov N.\/} Do it yourself structure constants for Lie algebras of types 
	$\E_l$ // Зап. научн. сем. ПОМИ. 2001. Т. 281. С. 60--104. 
	
	\bibitem{V19} {\it Vavilov N.\/} Reshaping the metaphor of proof // Philos. Trans. Roy. Soc. A. 2019.
	Vol. 377, №~2140. P. 1--18. doi: 10.1098/rsta.2018.0279
	
	\bibitem{Wa} {\it van der Waerden B.~L.} Moderne Algebra.  
	Berlin---G\"ottingen---Heidelberg: Springer-Verlag, 1931. %% 224 pp.
	
	\bibitem{W} {\it Weber H.} Lehrbuch der Algebra. Braunschweig: F.~Vieweg und Sohn, 
	Bd.~I, 1895, Bd.~II, 1896.
	
	
	\bibitem{W88} {\it Wolfram S.\/} {\tt Mathematica}: A System for Doing Mathematics by Computer. 
	Boston: Addison-Wesley Publishing Company,  2003. 1488 p. 
	
	\bibitem{W02} {\it Wolfram S.\/} A new kind of science // Wolfram Media, Inc., 2002. [Online]. URL: \url{https://www.wolframscience.com/nks/} (дата обращения: 22.06.2020).
	
	\bibitem{W20} {\it Wolfram S.\/} Finally we may have a path to the fundamental theory 
	of physics and it’s beautiful/ [News] 2020 April 14, \url{https://writings.stephenwolfram.com/2020/04/} (дата обращения: 22.06.2020).
	
	\bibitem{Ze} {\it Zeilberger D.\/} Theorems for a price: tomorrow's semi-rigorous mathematical 
	culture // Notices Amer. Math. Soc.  1993. Vol. 40, №~8. C. 978--981; reprinted in
	Math. Intelligencer. 1994. Vol. 16, №~4. P. 11--14.}

	
\end{thebibliography}

\end{document}